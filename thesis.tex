% ----------------------------------------------------------------------
%                   LATEX TEMPLATE FOR PhD THESIS
% ----------------------------------------------------------------------

% based on Harish Bhanderi's PhD/MPhil template, then Uni Cambridge
% http://www-h.eng.cam.ac.uk/help/tpl/textprocessing/ThesisStyle/
% corrected and extended in 2007 by Jakob Suckale, then MPI-CBG PhD programme
% and made available through OpenWetWare.org - the free biology wiki
% and finally modified in 2015-2017 by Holger Nahrstaedt
% https://github.com/holgern/TUB_PhDThesisTemplate

%: Style file for Latex
% Most style definitions are in the external file PhDthesisTUB.
% In this template package, it can be found in ./Classes/

\documentclass[twoside,11pt,online,a4paper,pdfa1,custommargin,numbered,biblatex]{Classes/PhDthesisTUB}
% *********************** Choosing pdfx standard ******************************
% `pdfa1'
% `pdfa2'
% `pdfx3'
% *********************** Change Thesis-Language to German ******************************
% default is english
% `german' : language is set to german
%
% % *********************** Choosing oneside / twoside ******************************
% `oneside' : layout is optimized for one-side print
% `twoside' : layout is optimized for two-side print
% *********************** Choosing print / online ******************************
% `print' : pdf-file is optimized for print
% `online' : pdf-file is optimized for online submission. The links are colorfull.
% *********************** Choosing biblatex or bibtex ******************
% `biblatex' : biblatex is used. Biblatex is automatically set when using xetex
%
% *********************** Choosing bibliographystyle ******************
% `numbered' : (default option) e.g. [1], [2]
% `authoryear' :  e.g. Name (2008)
% `custombib' : Use your own style, which is defined in preamble.tex
% *********************** Choosing the Fonts size ******************
% `9pt'
% `10pt'
% `11pt'
% `12pt'
% *********************** Choosing the paper size ******************
% `letterpaper'
% `a4paper'
% `a5paper'
% *********************** Choosing the Fonts in Class Options when using pdflatex ******************
% 
%  On Windows, the packge cm-super has to be installed!
%
% `' :  computer modern
% `fontA' :  newtxmath
% `fontB' :  cmbright
% `fontC' :  libertine,newtxmath
% `fontD' :  concmath
% `fontE' :  iwona
% `fontF' :  kurier	
% `fontG' :  anttor
% `fontH' :  kmath,kerkis
% `fontI' :   mathdesign (Utopia)
% `fontJ' :  fouriernc
% `fontK' :  pxfonts
% `fontL' :  mathpazo
% `fontM' :  mathpple
% `fontN' :  txfonts
% `fontO' :  mathtime (Belleek)
% `fontP' :  mathptmx	times	
% `fontQ' :  mbtimes	omega
% `fontR' :  arev
% `fontS' :  mathdesign (Charter)	
% `fontT' :  comicsans
% `fontU' :  mathdesign (Garamond)
% `fontV' :  fourier	utopia
% `fontW' :  ccfonts,eulervm
%
% `customfont': Use `customfont' option in the document class and load the
% package in the preamble.tex
%
% default or leave empty: `Latin Modern' font will be loaded.
%
% *********************** Choosing the Fonts in Class Options when using xelatex/lualatex ***********
%
% `' :  computer modern
% `fontA' :  XITS - XITS Math
% `fontB': Cambria - Cambria Math
% 'fontC': Libertinus - Libertinus Math
% 'fontD': TeX Gyre Pagella - Asana Math
% 'fontE': TeX Gyre Pagella - TeX Gyre Pagella Math
% 'fontF': TeX Gyre Schola - TeX Gyre Schola Math
% 'fontG': TeX Gyre Termes - TeX Gyre Termes Math
% 'fontH': TeX Gyre Bonum - TeX Gyre Bonum Math
% 'fontI': DejaVu Sans - TeX Gyre DejaVu Math
%
% `customfont': Use `customfont' option in the document class and load the
% package in the preamble.tex
%
% default or leave empty: `Latin Modern' font will be loaded.
%
% ************************* Custom Page Margins ********************************
%
% `custommargin`: Use `custommargin' in options to activate custom page margins,
% which can be defined in the preamble.tex. Custom margin will override
% print/online margin setup.
%
% ************************* other options ********************************
% `abstract`: Only the title-page and the abstracts are generated
%

\include{Preamble/preamble}

\title{An Extensible Framework for Integrating Multifaceted Transparency Enhancing Technologies in Cloud Native Systems}
% subtitle can be let empty
\subtitle{}
%\subtitle{}



% ----------------------------------------------------------------------
% The section below defines www links/email for author and institutions
% They will appear on the title page of the PDF and can be clicked
\ifpdf
  % The crest is a graphics file of the logo of your research institution.
  % Place it in ./0_frontmatter/figures and specify the width
  \crest{}
% If you are not creating a PDF then use the following. The default is PDF.
\else
%  \crest{\includegraphics[width=4cm]{logo.png}}
  \crest{}
\fi
  \author{Paskal Paesler}
\collegeordept{von der Fakult\"at IV - Elektrotechnik und Informatik Fachgebiet Information Systems Engineering}
\university{der Technischen Universit\"at Berlin}
\degreeplaceyear{Berlin 2022}
% set  Vorsitzender/Vorsitzende, Gutachter/Gutachterin
\comiteeheadidentifier{Vorsitzender}
\firstrevieweridentifier{Gutachter}
\secondrevieweridentifier{Gutachterin}
\thirdrevieweridentifier{Gutachter}
\forthrevieweridentifier{Gutachter}
\fifthrevieweridentifier{Gutachter}
% needed
\comiteehead{Prof. A}
\firstreviewer{Prof. B}
% can be let empty 
\secondreviewer{Prof. C}
\thirdreviewer{Prof. D}
\forthreviewer{}
\fifthreviewer{}

% TODO: matrikel nr., email, "vorgelegt von weg", 2. gutachter, 

%: ----------------------- set languange ------------------------
\ifCLASSINFOlangDE
\selectlanguage{german}
\else
\selectlanguage{english}
\fi
% ***************************** Abstract Separate ******************************
% To printout only the titlepage and the abstract with the PhD title and the
% author name for submission to the Student Registry, use the `abstract' option in
% the document class.

\ifdefineAbstract
 \pagestyle{empty}
 \includeonly{0_frontmatter/zusammenfassung, 0_frontmatter/abstract}
\fi

%: ----------------------- generate glossary ------------------------
\begin{document}

%: ----------------------- generate cover page ------------------------
\frontmatter
% \maketitle generates a title page for the final submission
% \makepretitle generates a title page for evaluation process
% title and author... can be set in thesis-info.tex
\phantomsection
\addcontentsline{toc}{chapter}{Title Page}
\maketitle
%\makepretitle

%: ----------------------- Choose spacing ------------------------
%\singlespacing
\onehalfspacing
%\doublespacing

%: ----------------------- abstract ------------------------

% Your institution may have specific regulations if you need an abstract and where it is to be placed in the document. The default here is just after title.

% -*- root: ../thesis.tex -*-
% Thesis Abstract -----------------------------------------------------
\selectlanguage{german}

\begin{zusammenfassung}        %this creates the heading for the abstract page
\addcontentsline{toc}{chapter}{Zusammenfassung}
Im Bereich der Digitalisierung gewinnen Cloud-native Architekturen immer mehr an Bedeutung. Obwohl diese Entwicklung viele Vorteile mit sich bringt, ist ein Nachteil die Nachverfolgung personenbezogener Daten im Rahmen von regulatorischen Rahmenbedingungen, wie der GDPR.
Diese Arbeit analysiert die Herausforderungen, denen sich Unternehmen im Umgang mit diesem Thema gegenübersehen, und bietet eine mögliche Lösung dafür.


Insbesondere wird das Hawk-Projekt erweitert und um neue Konzepte ergänzt. Außerdem wird ein tiefer Einblick in Data Loss Prevention-Technologien gegeben. AWS Macie und Google Cloud DLP werden verglichen und eingehend erläutert.
\end{zusammenfassung}
\ifCLASSINFOlangDE
\selectlanguage{german}
\else
\selectlanguage{english}
\fi
% ---------------------------------------------------------------------- 


% Thesis Abstract -----------------------------------------------------
\ifCLASSINFOlangDE
\selectlanguage{english}
\fi

%\begin{abstractslong}    %uncommenting this line, gives a different abstract heading
\begin{abstracts}        %this creates the heading for the abstract page
\addcontentsline{toc}{chapter}{Abstract}

In the realm of digitalization, cloud-native architectures are becoming more and more prominent. Although this development has many advantages, one disadvantage is the tracking of personal data in the scope of regulatory frameworks, such as the GDPR.
This thesis analyzes the challenges companies are facing when dealing with this topic and offers a possible solution for it.


More specifically, this thesis extends to the Hawk Project and adds new concepts. Also, a deep dive into Data Loss Prevention technologies will be presented. AWS Macie and Google Cloud DLP will be compared and explained in depth.

\end{abstracts}
%\end{abstractlongs}
\ifCLASSINFOlangDE
\selectlanguage{german}
\fi

% ---------------------------------------------------------------------- 



%: ----------------------- tie in front matter ------------------------

%\frontmatter
% -*- root: ../thesis.tex -*-
% Thesis Abstract -----------------------------------------------------
\selectlanguage{german}

\begin{erklärung}        %this creates the heading for the abstract page
\addcontentsline{toc}{chapter}{Erklärung zur Verfassung der Arbeit}
Hiermit erkläre ich, dass ich die vorliegende Arbeit selbstständig und eigenhändig sowie ohne unerlaubte fremde Hilfe und ausschließlich unter Verwendung der aufgeführten Quellen und Hilfsmittel angefertigt habe.


\vspace{25mm}

\begin{minipage}[t]{5cm}
\flushleft
\hrulefill \\
Paskal Paesler
\end{minipage}
\hfill
\begin{minipage}[t]{7cm}
\flushright
Berlin, den 18. Januar 2023
\end{minipage}
\end{erklärung}
\ifCLASSINFOlangDE
\selectlanguage{german}
\else
\selectlanguage{english}
\fi
% ---------------------------------------------------------------------- 

% Thesis Acknowledgements ------------------------------------------------


%\begin{acknowledgementslong} %uncommenting this line, gives a different acknowledgements heading
\begin{acknowledgements}      %this creates the heading for the acknowlegments

I would like to thank my advisors Elias Grünewalde and Karl Wolf for their constant support and assistance while working on this thesis. I would also like to thank Holger Nahrstaedt for providing this template: \url{https://github.com/holgern/TUB_PhDThesisTemplate}


\end{acknowledgements}
%\end{acknowledgmentslong}

% ------------------------------------------------------------------------





%: ----------------------- contents ------------------------
\tableofcontents            % print the table of contents

%: ----------------------- list of figures/tables ------------------------
\cleardoublepage
\listoffigures	% print list of figures
\cleardoublepage
\listoftables  % print list of tables


%: --------------------------------------------------------------
%:                  MAIN DOCUMENT SECTION
% --------------------------------------------------------------

\mainmatter

%: ----------------------- subdocuments ------------------------

% Parts of the thesis are included below. Rename the files as required.
% But take care that the paths match. You can also change the order of appearance by moving the include commands.
% \cfchapter[short name] {full name} {folder name} {file name}.
\cfchapter{Introduction}{1_introduction}{introduction}
\cfchapter{Problem}{2}{problem}
\cfchapter{Background and Related Work}{3}{background}
\cfchapter{Transparency at Runtime}{4}{runtime}
\cfchapter{Requirements}{5}{requirements}
\cfchapter{General Approach}{6}{approach}
\cfchapter{Implementation}{7}{implementation}
\cfchapter{Evaluation \& Discussion}{8}{evaluation}
\cfchapter{Conclusion}{9}{conclusion}
\cleardoublepage
       % description of lab methods



% --------------------------------------------------------------
%:                  BACK MATTER: appendices, refs,..
% --------------------------------------------------------------

% the back matter: appendix and references close the thesis


%: ----------------------- bibliography ------------------------

% The section below defines how references are listed and formatted
% The default below is one column, small font, complete author names.
% Entries are also linked back to the page number in the text and to external URL if provided in the BibTex file.

\begin{footnotesize} % tiny(5) < scriptsize(7) < footnotesize(8) < small (9)

\ifCLASSINFOlangDE
%\renewcommand{\bibname}{References}
\else
\renewcommand{\bibname}{References} % changes the header; default: Bibliography
\fi
% add bib to toc
\cleardoublepage
\phantomsection
\addcontentsline{toc}{chapter}{\bibname}

\ifCLASSINFObiblatex
\printbibliography
\else
% PhDbiblio-url2 = names small caps, title bold & hyperlinked, link to page 
\bibliographystyle{Classes/PhDbiblio-url2} % Title is link if provided
%\bibliographystyle{apalike}
\bibliography{backmatter/references} % adjust this to fit your BibTex file
\fi
\end{footnotesize}

\end{document}
