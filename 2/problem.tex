% -*- root: ../thesis.tex -*-
%!TEX root = ../thesis.tex
% ******************************* Thesis Chapter 2 ****************************


% ----------------------- paths to graphics ------------------------

% change according to folder and file names
\graphicspath{{2/figures/}}
% ----------------------- contents from here ------------------------

As stated above, the core problem is the relatively low usage of transparency-enhancing technologies. Below is a list of  possible reasons: \\

\noindent\textbf{No single technology can draw the complete picture} Meaning no single technology can solve the purpose of identifying and inspecting the personal data flow to a point, that, e.g., a data protection impact assessment can be generated.\\

\noindent\textbf{Missing abstraction layer} There are different technologies available, that solve the same purpose but have completely different APIs, which are hard to interchange. This is especially problematic for closed-source technologies, which create a vendor lock-in effect using this practice.\\

\noindent\textbf{Implementation effort} A problem is the often low cost-benefit ratio. For example, with DLP, both a way to forward the data that should be analyzed to the DLP tool and a way to display the results of that DLP tool afterward must be found and in many cases custom implemented. This then "only" helps to report possible security/privacy flaws. For example, \parencite[Art. 25]{noauthor_general_2016} says, that only measures with "reasonable" implementation costs should be realized

% ---------------------------------------------------------------------------
% ----------------------- end of thesis sub-document ------------------------
% ---------------------------------------------------------------------------